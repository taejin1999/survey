
\chapter{Conclusions}
\label{chap:Conclusions}

\section{Summary and Conclusions}


The cost-per-bit of NAND flash-based solid-state drives (i.e., SSDs) has steadily improved through uninterrupted semiconductor process scaling and multi-leveling so that they are how widely employed in not only mobile embedded systems but also personal computing systems.
However, the limited lifetime of NAND flash memory, as a side effect of recent advanced device technologies, is emerging as one of the major concerns for recent high-performance SSDs, especially for datacenter applications.


In this dissertation, we proposed several cross-layer optimization techniques to improve the lifetime (particularly endurance) of NAND flash memory.
Although the performance and reliability requirements of NAND flash memory are designed under the worst-case operating conditions of a storage product, the maximum capabilities of NAND devices are not fully utilized in most cases.
This observation has motivated us to propose a versatile device-level framework (i.e., DeVTS), including a NAND endurance model and newly defined device setting interfaces, that allows a flash software to exploit the tradeoffs between the endurance and performance/retention capabilities of NAND flash memory.


We have developed several SSD lifetime improvement techniques based on the DeVTS framework that supports various erase scaling modes and write capability tuning modes, each of which has a different impact on NAND endurance.
By accurately predicting the NAND requirements of write requests, our proposed techniques optimally tune the performance and retention capabilities of NAND devices.
%Since the proposed techniques can efficiently adapt to varying characteristics of I/O workloads, NAND endurance is significantly improved while maintaining the other NAND requirements such as performance and retention.
We have implemented dvsFTL, based on the DeVTS framework and proposed lifetime improvement techniques, that dynamically selects erase voltage/time scaling modes and write performance/retention capability tuning modes depending on varying workload conditions.
The existing garbage collector and wear leveler are also redesigned to maximize the efficiency of dvsFTL.
Since the performance and retention capabilities of NAND devices are frequently relaxed, dvsFTL manages the NAND requirements in a reliable fashion.


In order to evaluate the effectiveness of the proposed lifetime improvement techniques, we have built a timing-accurate NAND simulation environment which accurately emulates temporal interactions between varying I/O requests and various NAND operations.
Our experimental results show that when the write-performance tuning technique is employed, NAND endurance is improved by 62\% on average.
When the retention-capability tuning technique is added to dvsFTL, NAND endurance is further improved by 94\%, on average, over an existing DeVTS-unaware FTL.
In our evaluation, the overall write performance and retention requirements of storage systems are reliably maintained.


Since our proposed lifetime improvement techniques aggressively tune down the retention capability of NAND flash memory, data loss may occur due to retention failures when power is suddenly cut off.
Consequently, we have suggested a new data recovery technique to recover corrupted data from retention failures by exploiting the unique retention loss mechanism of NAND flash memory.
Our experimental results show that our proposed data recovery technique can recover from retention failures up to 23x faster over the existing recovery technique.
Furthermore, it effectively extends the NAND retention time by up to 8x over the specified retention time.


Since the proposed lifetime improvement techniques and reliability management techniques require only a small resource overhead and a negligible time overhead, they can easily be implemented into the existing NAND flash-based storage systems with minimal changes in flash software modules.


\section{Future Work}


\subsection{Lifetime Improvement Technique Exploiting\\ The Other NAND Tradeoffs}

The lifetime improvement techniques in this dissertation take advantage of variations in the write performance and retention requirements.
%, thus conditionally relaxing the performance and retention capabilities of NAND flash memory.
However, if variations in the maximum required number of read counts for each NAND page is additionally exploited, NAND endurance can be further improved.
%the erase voltage can be further lowered, thus maximizing the lifetime benefit of our proposed techniques.
%For example, the maximum read count of an MLC NAND block is up to 1,000K~\cite{ReadDisturb_Ha}.
%When the specified maximum read count for a write request is reduced to less than 100, the \textit{Vth} window can be further saved by about 500 mV.
For example, if the maximum read count of an MLC NAND block is reduced from 1,000K~\cite{ReadDisturb_Ha} to 1,000, the \textit{Vth} window can be additionally saved by about 500 mV.
Since the saved \textit{Vth} window by retention-capability tuning is about 500 mV, the effect of read-disturb resistance tuning on improving NAND endurance will be comparable to that of retention-capability tuning.

However, unlike the performance and retention capability tuning techniques, there is a challenging issue in the implementation of a read-disturb resistance tuning technique.
%unlike write performance/retention capability tuning, implementing read-disturb resistance tuning in real environments is a more complex technical challenges.
Since the proposed performance and retention capability tuning techniques exploit the spatial and temporal locality of write requests, it is possible to accurately predict the characteristics of the near-future write requests.
On the contrary, it is difficult to predict the future read intensity of the current write request in a storage software layer.
In order to decide whether or not the read-disturb resistance of the current write request can be relaxed, it is necessary to exploit more higher-level hints from file systems or applications.
If such useful information for the future read intensity of write requests can be exploited, the endurance gain of the proposed techniques is maximized.


%\subsection{Application to DRAM-Flash Hybrid Main\\ Memory Systems}
\subsection{Development of Extended Techniques for\\ DRAM-Flash Hybrid Main Memory Systems}

As big data analytics based on massive data, rapidly generated and processed, become commonplace in real environments, there is a strong demand on high-performance computing systems that can efficiently store and process such massive data in real time.
The most critical requirement on the next-generation information systems, such as intelligent self-driving control systems, based on the big data analytics is to keep extremely high performance in a consistent manner.
In order to satisfy such a requirement, most of existing optimization techniques have mainly focused on \textit{in-memory processing} that can prevent from accessing to slow storage systems.
The existing DRAM-based main memory system, however, is not a practical solution for such big data applications because of its pool cost/energy efficiencies.
In order to implement a cost-efficient main memory system with a huge capacity as well as low power consumption, several system-level approaches have been suggested by taking advantage of both DRAM and NAND flash memory through a new software architecture~\cite{SSDAlloc_Badam} or hardware architecture~\cite{FlashMainMemory_Jacob}.
However, the limited lifetime of NAND flash memory can be also a serious reliability issue when such DRAM-Flash hybrid main memory systems are actively employed in real environments.


If the operating systems can directly manage the proposed lifetime improvement techniques by exploiting various new interactions between DRAM and NAND flash under an NVDIMM-like setting, it is possible to extend the lifetime of NAND flash to the fullest extent.
%If the proposed lifetime improvement techniques can be directly managed by the operating systems, it is possible to extend the lifetime of NAND flash memory to the fullest extent.
For example, by exploiting many useful hints, disappeared while passing through I/O stacks, in the host system, the performance and retention requirements of the requests can be more reliably and directly classified.
Moreover, since our proposed techniques can easily be combined with existing data reduction techniques such as data compression and data de-duplication, NAND lifetime can be further extended.
Our proposed lifetime improvement techniques can be a crucial breakthrough in the new type of main memory systems.


\subsection{Development of Specialized SSDs}

Recently, in order to optimally exploit the unique superiorities (e.g., non-volatility, high write throughput, and low access latency) of NAND devices, several types of specialized SSDs are required in datacenter environments~\cite{Flash_Facebook}.
For example, when SSDs are used as cache, lower latency as well as higher endurance is needed.
On the other hand, when SSDs are used as a {\it cold storage}, a higher capacity with a longer retention time is more preferable.
However, existing SSD products do not fulfil such various requirements in a single device because most of capabilities of NAND flash memory usually fixed during device design times.
In order to meet such requirements from datacenter applications, it is required to develop a multi-purpose SSD whose capabilities can be flexibly adjusted on demand.


The primary goal of this dissertation is to improve NAND endurance by conditionally tuning down the other NAND capabilities.
In order to achieve this research goal, we propose the NAND endurance model which accurately captures the tradeoff relationship among the NAND capabilities.
Since the relationship between each NAND capability is expressed as the saved \textit{Vth} window by each tuning technique, the proposed NAND endurance model can be utilized for other purpose such as booting the write performance or retention capability of a storage device.
For example, when urgent write requests are issued to a storage system, the write performance of NAND devices can be rapidly boosted by temporarily sacrificing the endurance and retention capabilities of NAND devices.
Similarly, when cold data are to be written, the retention capability of NAND devices can be further enhanced by sacrificing the write performance of NAND devices.
%To the best of our knowledge, it is difficult to make such decisions within a storage devices.
%In order to deal with rapidly changing objects in a reliable fashion, it is more efficient to make a decision in the file systems.


%\subsection{Development of A Framework for Managing NAND Reliability Requirements}



