
\chapter{Conclusions}
\label{chap:Conclusions}

\section{Summary and Conclusions}
The cost-per-bit of NAND flash-based solid-state drives (i.e., SSDs) has 
steadily improved through uninterrupted semiconductor process scaling and 
multi-leveling so that they are how widely employed in not only mobile embedded 
systems but also personal computing systems.
However, the limited lifetime of NAND flash memory, as a side effect of recent 
advanced device technologies, is emerging as one of the major concerns for 
recent high-performance SSDs, especially for datacenter applications.

In this dissertation, we proposed several system-level techniques that improve
the lifetime of NAND flash-based storage devices. We first presented
data separation technique, called PCStream, for multi-streamed SSDs.
Unlike existing stream management techniques, \textsf{\small PCStream} fully automates 
the process of mapping data to a stream based on PCs, 
which work well for append-only workloads as well as update workloads.  
By exploiting an observation that most PCs are distinguishable from each other 
in their lifetime characteristics, \textsf{\small PCStream} allocates each PC to a different stream.  
When a PC has a large variance in their lifetimes, \textsf{\small PCStream} refines its stream allocation 
during garbage collection and moves the long-lived data of the current stream to its substream.  

Next, 
we propose a fine-grained deduplication technique for flash-based SSDs, called FineDedup.
By using a fine-grained deduplication unit,
the proposed FineDedup technique increases the amount of data eliminated 
by data deduplication by up to 37\% over the existing page-based deduplication technique,
extending the SSD lifetime by the same amount.
FineDedup inevitably increases the overall read response time because of data fragmentation.
By employing a chunk read buffer and a chunk packing scheme,
however, the read performance overhead is limited to less than 5\% 
in comparison with the existing deduplication technique.
To reduce the memory space required for a chunk-level mapping table,
FineDedup adopts a hybrid mapping scheme.
Our evaluation results show that 
FineDedup is effective in improving the SSD lifetime,
requiring only about 10 MBs of more memory space in total.

\section{Future Work}
\subsection{Improving stream mapping method of PCStream}

The current version of \textsf{\small PCStream} can be extended in several directions.  
For example, we plan to optimize the PC clustering method so that
multiple PCs can be better clustered when the number of PCs significantly
outnumbers the number of streams.  
For example, \textsf{\small PCStream} should be improved in its PC clustering method 
so that it can work effectively even when there are more PCs than the number of streams.  
We also plan to evaluate \textsf{\small PCStream} on real SSDs 
by implementing the two-phase stream assignment algorithm inside an FTL.
