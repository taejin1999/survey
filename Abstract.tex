\begin{abstract}

Replacing HDDs with NAND flash-based storage devices (SSDs) has been one of the major 
challenges in modern computing systems especially in regards to better performance and higher mobility.
Although the continuous semiconductor process scaling and multi-leveling techniques 
lower the price of SSDs to the comparable level of HDDs, the decreasing lifetime of NAND flash memory, 
as a side effect of recent advanced device technologies, 
is emerging as one of the major barriers to the wide adoption of SSDs in high-performance computing systems.

In this dissertation, system-level lifetime improvement techniques for
recent high-density NAND flash memory are proposed.
Unlike existing techniques, the proposed techniques resolve the
problems of decreasing lifetime by exploiting the write request characteristics,
such as context of I/O or duplicate data contents.

We first propose a system-level approach to reduce WAF that exploits
the I/O context of an application to increase the data lifetime prediction
for the multi-streamed SSDs. 
Since the key motivation behind the proposed technqiue was 
that data lifetimes should be estimated at a higher abstraction level than LBAs, 
we employ a write program context as a stream management unit.
Thus, it can effectively separate data with
short lifetimes from data with long lifetimes to improve the efficiency of garbage collection.

Second, we present a write traffic reduction approach which reduces the amount of
write traffic sent to a storage by eliminating redundant data, therby improving
the lifetime of storage devices. 
In particular, we improve the likelihood of eliminating redundant data
by introducing sub-page chunk based on the understanding of duplicate contents
such as partial updates or zero padding of file system.
It also resolves technical difficulties caused by its finer granularity, i.e., increased memory requirement and read
response time. 

In order to evaluate the effectiveness of the proposed techniques, we performed 
a series of evaluations using both a trace-driven simulator and emulator with I/O
traces which were collected from various real-world systems.
To understand the feasibility of the proposed techniques, we also implemented them
in Linux kernel on top of our in-house flash storage prototype and then evaluated
their effects on the lifetime while running real-world applications.
Our experimental results show that system-level optimization techniques are
more effective over existing optimization techniques.

\end{abstract}
