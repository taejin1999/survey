
\chapter{Introduction} 
\label{chap:Introduction}

\section{Motivation}
\label{sec:Intro_Motivation}

NAND flash-based solid-state drives (SSDs) are widely used in personal computing systems as well as mobile embedded systems.
However, in enterprise environments, SSDs are employed in only limited applications because SSDs are not yet cost competitive with HDDs~\cite{Janus_albrecht}.
Fortunately, the prices for SSDs have fallen to the comparable level of HDDs by continuous semiconductor process scaling (e.g., 10 nm-node process~\cite{15nmMLC_Sako}) combined with multi-leveling technologies (e.g., MLC~\cite{21nmMLC_Kim} and TLC~\cite{TLC_Shin}).
However, the limited endurance of NAND flash memory, which have declined further as a side effect of the recent advanced device technologies, is emerging as another major barrier to the wide adoption of SSDs.
(\textit{NAND endurance} is the ability of a memory cell to endure program/erase (P/E) cycling, and is quantified as \textit{the maximum number $N_{P/E}^{max}$ of P/E cycles} that the cell can tolerate while maintaining its reliability requirements~\cite{Flash_Brewer}.)
For example, although the NAND capacity per die doubles every two years, the actual lifetime (which is proportional to the total NAND capacity and $N_{P/E}^{max}$) of SSDs does not increase as much as projected in the past seven years because $N_{P/E}^{max}$ have declined by 70\% during that period~\cite{MooresLaw_chien}.
In order for SSDs to be commonplace in enterprise environments, the issues concerning NAND endurance should be properly resolved.

Since the Lifetime $L_{C}$ of an SSD with the total capacity $C$ is proportional to the maximum number $N_{P/E}^{max}$ of P/E cycles, and is inversely proportional to the total written data $W_{day}$ per day, $L_{C}$ (in days) can be expressed as follows (assuming a perfect wear leveling):
\begin{equation}\label{eq:introduction_1}
L_{C} = \frac{N_{P/E}^{max}\: \times \: C}{W_{day}\: \times \: WAF} \quad ,
\end{equation}
where $WAF$ is a write amplification factor which represents the efficiency of an FTL algorithm.
Many existing lifetime-enhancing techniques have mainly focused on reducing $WAF$ by increasing the efficiency of an FTL algorithm.
For example, by avoiding unnecessary data copies during garbage collection, $WAF$ can be reduced~\cite{HotCold_Hsieh}.
In order to reduce $W_{day}$, various system-level techniques were proposed.
For example, data de-duplication~\cite{CAFTL_Chen}, data compression~\cite{Compression_Lee}, and write traffic throttling~\cite{DT_Lee} are such techniques.
On the other hand, only a few system/software-level techniques have been proposed to increase $N_{P/E}^{max}$.
Although several conceptual device-level techniques (e.g., a self-healing SSD~\cite{SelfHealing_Wu}) were suggested regarding $N_{P/E}^{max}$, it is difficult for these to be employed in real systems because of their unrealistic hardware settings and critical side-effects.

By exploiting the tradeoff relationships between the NAND characteristics (e.g., capacity, performance, retention, and endurance), several cross-layer optimization techniques have been suggested.
In order to improve SSD performance, for example, the retention relaxation technique~\cite{RetentionRelaxation_Liu} temporarily relaxes the NAND retention capability while FlexFS~\cite{Flexfs_Lee} flexibly reorganizes the NAND capacity between SLC and MLC regions.
Although these techniques exploited the device-level physical characteristics in the similar fashion of our work, their main goals are quite different from ours.
Up until now, there have been a few particular suggestions to improve the NAND endurance by exploiting the tradeoff relationships between the NAND capabilities.


\section{Dissertation Goals}
\label{sec:Intro_ResearchGoals}

In this dissertation, we propose new cross-layer optimization techniques to extend the lifetime of NAND flash-based storage devices by exploiting the tradeoff relationship among NAND capabilities such as endurance, performance, and retention.
%Our proposed techniques enables a system software to directly exploit the tradeoff relationship between the NAND endurance and NAND operating voltages/times so that the NAND requirements, such and endurance, performance, and retention, can be properly managed.
The primary goals of this dissertation is as follows:

\begin{itemize}
\item Enabling a system software to exploit the tradeoff relationship between the endurance and the other capabilities of NAND flash memory.
\item Developing system-level techniques to improve NAND endurance while maintaining the other NAND requirements.
\item Providing reliability preservation techniques for NAND flash-based storage systems when flash-optimization techniques are widely employed in real environments.
\end{itemize}


\section{Contributions}
\label{sec:Intro_Contributions}

The proposed cross-layer approach in this dissertation adds a new dimension to the decreasing lifetime problem of NAND flash-based storage devices as follows:

\begin{itemize}
\item {\bfseries A unified NAND endurance model} which captures the tradeoff relationship between NAND endurance and the performance/retention capabilities of NAND flash memory is proposed.
We reveal that endurance degradation is primarily caused by excessive erase operations, and suggest effective device-level means (i.e., various write-capability tuning techniques) of alleviating the negative impact of erase operations on NAND endurance.
Based on the proposed NAND endurance model, a system software can adjust the internal operation voltages and times of NAND flash memory in a reliable fashion.
\item {\bfseries System-level lifetime improvement techniques} for NAND flash-based storage devices are presented.
Based on the NAND endurance model, the proposed techniques dynamically change the NAND performance and retention capabilities for each program operation so that endurance-enhancing erase operations can be frequently used.
Since the proposed lifetime improvement techniques can efficiently adapt to varying characteristics of I/O workload by accurately predicting the write performance and retention requirements, the overall performance and reliability requirements of storage systems are maintained while significantly improving NAND endurance.
%select the erase scaling modes and write tuning modes depending on varying workload conditions.
\item {\bfseries Reliability management techniques} for NAND flash-based storage systems are suggested.
Since the proposed lifetime improvement techniques aggressively tune down the NAND retention capability to improve NAND endurance, the retention-failure problem can be a serious technical issue for power/temperature-unstable computing environments.
In order to preserve the data durability of the stored data in NAND flash memory, we introduce a novel data recovery technique which can efficiently and quickly recover corrupted data from retention failures.
\end{itemize}

Although this dissertation has mainly focused on improving NAND endurance, our proposed techniques can be extended to improve other requirements (e.g., performance, retention, and read-disturbs resistance) of storage systems.
Moreover, since our techniques are entirely independent on data content, the existing flash-optimization techniques can be easily integrated into our proposed framework.


\section{Dissertation Structure}
\label{sec:Intro_DissertationStructure}


%The rest of the paper is organized as follows.
%Before our proposed techniques are presented, we briefly review the design principle of NAND flash memory in Section~\ref{sec:Background} and the previously proposed lifetime-enhancing techniques in Section~\ref{sec:RelatedWorks}.
%Section~\ref{sec:DynamicEraseVoltageTimeScaling} describes the proposed erase voltage and time scaling technique.
%In Section~\ref{sec:DynamicWriteModeTuning}, the dynamic write-tuning techniques and the NAND endurance model are presented.
%Our autoFTL with dynamic write-speed tuning modes is proposed in Sections~\ref{sec:autoFTL}, and experimental results follow in Section~\ref{sec:ExperimentalResults}.
%Section~\ref{sec:autoFTLplus.tex} explains our ongoing works for designing autoFTL$^{+}$ with dynamic retention tuning and its initial experimental results.
%Finally, Section~\ref{sec:Conclusion} concludes with a summary and future work.

This dissertation consists of seven chapters.
The first chapter presents a introduction to this dissertation while the last chapter serves as a conclusion with a summary and future work.
The five intermediate chapters are organized as follows:

Chapter~\ref{chap:Background} reviews the operational principles of NAND flash memory and explains existing SSD lifetime improvement techniques closely related to this dissertation.

Chapter~\ref{chap:DynamicEraseVoltageTimeScaling} describes the dynamic NAND voltage and time scaling framework which includes erase voltage/time scaling and write capability tuning.
Combining erase scaling and write tuning, a unified NAND endurance model for estimating their effects on NAND endurance is also suggested.

Chapter~\ref{chap:LifetimeImprovementWPT} proposes an SSD lifetime improvement technique using write-performance tuning.
%We explain how to select appropriate erase and write modes and show how much NAND endurance is improved.
We explain how to use a lower voltage and a slower speed for an erase operation and how to write data to a NAND block erased with a lower voltage.
In addition, the effect of the proposed technique on NAND endurance is presented in detail.

Chapter~\ref{chap:LifetimeImprovementRCT} presents a comprehensive SSD lifetime improvement technique using both write-performance tuning and retention-capability tuning.
We describe reliable prediction schemes to accurately predict the write performance and retention requirement and present efficient adaptation schemes to manage the NAND capabilities.
We then show how much NAND endurance is improved and whether the overall NAND requirements are preserved.

Chapter~\ref{chap:ReliabilityPreservationTechnique} suggests a reliability management technique in order to recover data loss due to retention failures.
Finally, we show how efficient the proposed technique is in terms of data recovery power and speed.

